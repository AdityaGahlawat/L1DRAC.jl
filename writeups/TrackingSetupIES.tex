\documentclass[11pt]{article}

\usepackage{amsmath,amssymb,amsthm}
\usepackage[margin=1in]{geometry}
\usepackage{hyperref}

% Theorem environments
\newtheorem{assumption}{Assumption}

\title{Tracking Realizable Trajectories via Incremental Exponential Stability}
\author{Aditya Gahlawat}
\date{}

\begin{document}

\maketitle

\section{Nonlinear System}

Consider the control-affine system
\[
\dot x = f(x) + G u .
\]
% 
Assume a state-feedback controller
\[
u = k(x),
\]
and define the closed-loop vector field
\[
F(x) := f(x) + G k(x).
\]

\begin{assumption}[Incremental Exponential Stability]
The system
\[
\dot z = F(z)
\]
is \textbf{incrementally exponentially stable (IES)}, i.e., there exist constants $c \ge 1$ and $\lambda > 0$ such that for any two solutions $z_1(t)$, $z_2(t)$,
\[
\| z_1(t) - z_2(t) \| \le c e^{-\lambda t} \| z_1(0) - z_2(0) \| .
\]
\end{assumption}

\section{Reference Trajectory and Feedback--Feedforward Structure}

Let $x_r(t)$ be a realizable reference trajectory generated by
\[
\dot x_r = f(x_r) + G u_r .
\]
Add and subtract $k(x_r)$:
\[
\dot x_r = F(x_r) + G\big(u_r - k(x_r)\big) .
\]
Defining
\[
v_r := u_r - k(x_r),
\]
the reference input admits a \textbf{feedback + feedforward decomposition}.
% 
We can thus use the following \textbf{feedback + feedforward control law} for the actual system:
\[
u = k(x) + v_r ,
\]
where $v_r$ is the \textbf{feedforward term induced by the reference trajectory $(x_r, u_r)$} as defined above.

This yields the closed-loop dynamics
\[
\dot x = F(x) + G v_r,
\qquad
\dot x_r = F(x_r) + G v_r .
\]

\section{Error Dynamics}

Define the tracking error
\[
e := x - x_r .
\]

Then
\[
\dot e = F(x) - F(x_r)
       = F(x_r + e) - F(x_r) .
\]

\subsection{Interpretation}

Since $x(t)$ and $x_r(t)$ satisfy the \textbf{same system}
\[
\dot z = F(z) + G v_r(t),
\]
and $\dot z = F(z)$ is incrementally exponentially stable, the error dynamics represent the distance between two trajectories of an IES system.

\section{Special Case: Linear Time-Invariant (LTI) Systems}

\subsection{LTI Plant}

Consider the LTI system
\[
\dot x = A x + B u ,
\]
with linear state-feedback
\[
u = K x ,
\]
and assume
\[
A + B K \quad \text{is Hurwitz}.
\]

Define
\[
F(x) := (A + B K) x .
\]

\subsection{Reference Trajectory and Feedback--Feedforward Structure}

Let the reference trajectory satisfy
\[
\dot x_r = A x_r + B u_r .
\]

Defining
\[
v_r := u_r - K x_r ,
\]
the reference input admits a \textbf{feedback + feedforward decomposition}.

We can thus use the following \textbf{feedback + feedforward control law} for the actual system:
\[
u = K x + v_r ,
\]
where $v_r$ is the \textbf{feedforward term induced by the reference trajectory $(x_r, u_r)$}.

This yields
\[
\dot x = (A + B K) x + B v_r,
\qquad
\dot x_r = (A + B K) x_r + B v_r .
\]

\subsection{Error Dynamics}

Define $e := x - x_r$. Then
\[
\dot e = (A + B K) e .
\]

\subsection{Interpretation (LTI)}

Since $A + B K$ is Hurwitz, the system is incrementally exponentially stable, and the tracking error satisfies
\[
\| e(t) \| \le c e^{-\lambda t} \| e(0) \| ,
\]
for some $c \ge 1$, $\lambda > 0$, independent of the reference input $u_r$.

\nocite{placeholder}
\bibliographystyle{plain}
\bibliography{TrackingSetupIES}

\end{document}
