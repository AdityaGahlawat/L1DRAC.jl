\documentclass[11pt]{article}

\usepackage{xcolor}
\usepackage{amsmath,amssymb,amsthm}
\usepackage[margin=1in]{geometry}
\usepackage{hyperref}

% Theorem environments
\newtheorem{assumption}{Assumption}

\title{Tracking Realizable Trajectories via Incremental Exponential Stability}
\author{Aditya Gahlawat}
\date{}

\begin{document}

\maketitle

\tableofcontents

\section{Nonlinear System}

Consider the control-affine system
\[
\dot x = f_{ol}(x) + G u .
\]
% 
Assume a state-feedback controller
\[
u = k(x),
\]
and define the closed-loop vector field
\[
f_{cl}(x) := f_{ol}(x) + G k(x).
\]

\begin{assumption}[Incremental Exponential Stability]
The system
\[
\dot z = f_{cl}(z)
\]
is \textbf{incrementally exponentially stable (IES)}, i.e., there exist constants $c \ge 1$ and $\lambda > 0$ such that for any two solutions $z_1(t)$, $z_2(t)$,
\[
\| z_1(t) - z_2(t) \| \le c e^{-\lambda t} \| z_1(0) - z_2(0) \| .
\]
\end{assumption}

\section{Reference Trajectory and Feedback--Feedforward Structure}

Let $x^\star(t)$ be a realizable reference trajectory generated by
\[
\dot x^\star = f_{ol}(x^\star) + G u^\star .
\]
Add and subtract $k(x^\star)$:
\[
\dot x^\star = f_{cl}(x^\star) + G\big(u^\star - k(x^\star)\big) .
\]
Defining
\[
v^\star := u^\star - k(x^\star),
\]
the reference input admits a \textbf{feedback + feedforward decomposition}.
% 
We can thus use the following \textbf{feedback + feedforward control law} for the actual system:
\[
u = k(x) + v^\star ,
\]
where $v^\star$ is the \textbf{feedforward term induced by the reference trajectory $(x^\star, u^\star)$} as defined above.

This yields the closed-loop dynamics
\[
\dot x = f_{cl}(x) + G v^\star,
\qquad
\dot x^\star = f_{cl}(x^\star) + G v^\star .
\]

\section{{\color{red} General use in $\mathcal{L}_1$-DRAC}}

As in the manuscript, we define 
\begin{align}\label{eqn:main_f}
       f(t,x) := & f_{cl}(x) + G v^\star(t) = f_{ol}(x) + G \left( k(x) + u^\star(t) - k\left(x^\star(t)\right) \right)
       \notag 
       \\         
\end{align}

For our use case, we need to find $\Delta_f$ such that 
\begin{align*}
       % \norm{f_{cl}(t,x)}
\end{align*}

\section{Error Dynamics}

Define the tracking error
\[
e := x - x^\star .
\]

Then
\[
\dot e = f_{cl}(x) - f_{cl}(x^\star)
       = f_{cl}(x^\star + e) - f_{cl}(x^\star) .
\]

\subsection{Interpretation}

Since $x(t)$ and $x^\star(t)$ satisfy the \textbf{same system}
\[
\dot z = f_{cl}(z) + G v^\star(t),
\]
and $\dot z = f_{cl}(z)$ is incrementally exponentially stable, the error dynamics represent the distance between two trajectories of an IES system.

\section{Special Case: Linear Time-Invariant (LTI) Systems}

\subsection{LTI Plant}

Consider the LTI system
\[
\dot x = A x + B u ,
\]
with linear state-feedback
\[
u = K x ,
\]
and assume
\[
A + B K \quad \text{is Hurwitz}.
\]

Define
\[
f_{cl}(x) := (A + B K) x .
\]

\subsection{Reference Trajectory and Feedback--Feedforward Structure}

Let the reference trajectory satisfy
\[
\dot x^\star = A x^\star + B u^\star .
\]

Defining
\[
v^\star := u^\star - K x^\star ,
\]
the reference input admits a \textbf{feedback + feedforward decomposition}.

We can thus use the following \textbf{feedback + feedforward control law} for the actual system:
\[
u = K x + v^\star ,
\]
where $v^\star$ is the \textbf{feedforward term induced by the reference trajectory $(x^\star, u^\star)$}.

This yields
\[
\dot x = (A + B K) x + B v^\star,
\qquad
\dot x^\star = (A + B K) x^\star + B v^\star .
\]

\subsection{Error Dynamics}

Define $e := x - x^\star$. Then
\[
\dot e = (A + B K) e .
\]

\subsection{Interpretation (LTI)}

Since $A + B K$ is Hurwitz, the system is incrementally exponentially stable, and the tracking error satisfies
\[
\| e(t) \| \le c e^{-\lambda t} \| e(0) \| ,
\]
for some $c \ge 1$, $\lambda > 0$, independent of the reference input $u^\star$.

\nocite{placeholder}
\bibliographystyle{plain}
\bibliography{TrackingSetupIES}

\end{document}
